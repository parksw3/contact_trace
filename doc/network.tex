\documentclass[12pt]{article}
\usepackage[top=1in,left=1in, right = 1in, footskip=1in]{geometry}

\usepackage{graphicx}
%\usepackage{adjustbox}

\newcommand{\eref}[1]{(\ref{eq:#1})}
\newcommand{\fref}[1]{Fig.~\ref{fig:#1}}
\newcommand{\Fref}[1]{Fig.~\ref{fig:#1}}
\newcommand{\sref}[1]{Sec.~\ref{#1}}
\newcommand{\frange}[2]{Fig.~\ref{fig:#1}--\ref{fig:#2}}
\newcommand{\tref}[1]{Table~\ref{tab:#1}}
\newcommand{\tlab}[1]{\label{tab:#1}}
\newcommand{\seminar}{SE\mbox{$^m$}I\mbox{$^n$}R}

\usepackage{amsthm}
\usepackage{amsmath}
\usepackage{amssymb}
\usepackage{amsfonts}

\usepackage{lineno}
\linenumbers

\usepackage[pdfencoding=auto, psdextra]{hyperref}

\usepackage{natbib}
\bibliographystyle{chicago}
\date{\today}

\usepackage{xspace}
\newcommand*{\ie}{i.e.\@\xspace}

\usepackage{color}

\newcommand{\Rx}[1]{\ensuremath{{\mathcal R}_{#1}}} 
\newcommand{\Ro}{\Rx{0}}
\newcommand{\RR}{\ensuremath{{\mathcal R}}}
\newcommand{\Rhat}{\ensuremath{{\hat\RR}}}
\newcommand{\tsub}[2]{#1_{{\textrm{\tiny #2}}}}

\newcommand{\comment}[3]{\textcolor{#1}{\textbf{[#2: }\textsl{#3}\textbf{]}}}
\newcommand{\jd}[1]{\comment{cyan}{JD}{#1}}
\newcommand{\swp}[1]{\comment{magenta}{SWP}{#1}}
\newcommand{\dc}[1]{\comment{blue}{DC}{#1}}
\newcommand{\hotcomment}[1]{\comment{red}{HOT}{#1}}

\newcommand{\jdnew}{\jd{NEW}}
\newcommand{\jddel}[1]{\jd{DELETE: #1}}

\begin{document}

\begin{flushleft}{
	\Large
	\textbf\newline{
		Inferring generation-interval distributions from contact-tracing data
	}
}
\newline
\\
Sang Woo Park\textsuperscript{1,*}
David Champredon\textsuperscript{2,3}
Jonathan Dushoff\textsuperscript{2,4}
\\

\bigskip
\textbf{1} Department of Mathematics \& Statistics, McMaster University, Hamilton, Ontario, Canada
\\
\textbf{2} Department of Biology, McMaster University, Hamilton, Ontario, Canada
\\
\textbf{3} Agent-Based Modelling Laboratory, York University, Toronto, Ontario, Canada
\\
\textbf{4} Michael G. DeGroote Institute for Infectious Disease Research, McMaster University, Hamilton, Ontario, Canada
\\
\bigskip

*parksw3@mcmaster.ca
\end{flushleft} 

\section*{Abstract}

Generation intervals, defined as the time between when an individual is infected and when that individual infects another person, link two key quantities that describe an epidemic: the reproductive number, $\RR$, and the rate of exponential growth, $r$.
Generation intervals are often measured through contact tracing by identifying who infected whom.
We study how observed intervals differ from ``intrinsic'' intervals that could be estimated by tracing individual-level infectiousness, and identify both spatial and temporal effects, including censoring (due to observation time), and the effects of susceptible depletion at various spatial scales. 
Early in an epidemic, we expect the variation in the observed generation intervals to be mainly driven by the censoring and the population structure near the source of disease spread; 
therefore, we predict that correcting observed intervals for the effect of temporal censoring but \emph{not} for spatial effects will provide a spatially informed ``effective'' generation-interval distribution, which will correctly link $r$ and $\RR$.
We develop and test statistical methods for temporal corrections of generation intervals,
and confirm our prediction using individual-based simulations on an empirical network.

\section*{Keywords}

Infectious disease modeling, generation interval, basic reproduction number, population structure, contact tracing

\pagebreak

\section{Introduction}

An epidemic can be characterized by the exponential growth rate, $r$, and the reproductive number, \RR.
The reproductive number, defined as the average number of secondary cases arising from a primary case, is of particular interest as it provides information about the final size of an epidemic \citep{anderson1991infectious, diekmann1990definition}.
However, estimating the reproductive number directly from disease natural history requires detailed knowledge which is not often available, particularly early in an outbreak \citep{dietz1993estimation}.
Instead, the reproductive number can be \emph{indirectly} estimated from the exponential growth rate, which can be estimated from incidence data \citep{chowell2003sars, mills2004transmissibility, nishiura2009transmission, nishiura2010pros, ma2014estimating}.
These two quantities are linked by generation-interval distributions \citep{wearing2005appropriate, svensson2007note, roberts2007model, wallinga2007generation, park2019practical}.

A generation interval is defined as the time between when a person becomes infected and when that person infects another person \citep{svensson2007note}.
Due to individual variation in infection time, the observed generation-interval distribution can change depending on when and how it is measured \citep{svensson2007note, kenah2008generation, nishiura2010time, champredon2015intrinsic}.
There are important distinctions to be made when estimating generation intervals: \emph{intrinsic} generation intervals measure the infectiousness of an infected individual,
while \emph{observed} generation intervals refer to the time between actual infection events.
Observed generation intervals in turn can be measured either \emph{forward} in time by looking at a cohort of individuals infected at the same time and asking when they infected others or \emph{backward} in time by looking at a cohort and asking when their infectors were infected \citep{champredon2015intrinsic}.
% When aggregated generation intervals are observed until a given time in an ongoing epidemic (often via contact tracing), we refer to these as \emph{censored} (or right-censored) intervals.

% Early in an epidemic, when depletion of susceptibles is negligible, we expect the forward generation-interval distribution to be similar to the intrinsic generation-interval distribution \citep{champredon2015intrinsic}.
% As an epidemic progresses, an infector is less likely to infect individuals later in time due to decrease in susceptibles, 
% and the distribution of forward generation intervals will have shorter mean than the intrinsic distribution.
While the intrinsic generation-interval distribution is often assumed to be fixed, the shape of the observed generation-interval distribution depends on when and how intervals are measured \citep{kenah2008generation, nishiura2010time, tomba2010some, champredon2015intrinsic, britton2019estimation}. 
Here, we briefly describe how backward generation-interval distributions change over time because backward measurements better reflect real outbreak scenarios where contact tracing is performed by identifying infectors of infected individuals. (see \cite{kenah2008generation, nishiura2010time, tomba2010some, champredon2015intrinsic} for discussions on forward generation intervals).
When an epidemic is growing exponentially, as often occurs near the beginning of an outbreak, the number of newly infected individuals will be large relative to the number infected earlier on. 
A susceptible individual is thus relatively more likely to be infected by a newly infected individual, 
and the distribution of backward generation intervals will be shorter than the intrinsic distribution \citep{nishiura2010time, champredon2015intrinsic, britton2019estimation}.
When an epidemic is subsiding, most infections are caused by the remaining infectors, rather than new infectors, and backward generation intervals will be longer than the intrinsic one \citep{nishiura2010time, tomba2010some, champredon2015intrinsic}.

In practice, generation intervals are often measured by aggregating available information from contact tracing. 
%% Whereas backward generation intervals represent infections that are one generation apart from a cohort of infectees, these aggregated generation intervals represent infections across multiple cohorts (i.e., the time of infection of the infectees can range anywhere from the beginning of an epidemic to the most recent infection).
While an epidemic is ongoing, these measurements are ``censored'': we don't know what happens after the observation time.
Thus, there is a bias towards observing shorter intervals, which are more likely to have concluded by the observation time.
In fact, censored observations can be thought of as an average of backward generation-interval distributions, over the cohorts which have been infected before the observation time (and weighted by cohort size).

Realized generation intervals are also affected by spatial structure.
In a structured population (that is, any population that does not mix homogeneously), susceptibility will tend to decrease more quickly in the neighbourhood of infected individuals than in the general population even in the early phase of an epidemic. 
This means that contacts made late in an individual's infection are more likely to be ineffective due to contacts that were made earlier (because the contactee may have been infected already) 
As a result, realized generation intervals will have shorter mean than the intrinsic generation-interval distribution.
This perspective allows us to reinterpret the finding of \cite{trapman2016inferring} that, given an intrinsic generation interval and an observed growth rate, the reproductive number on various network structures is always smaller on a network than would be predicted from homogeneous mixing.
These observations allow us to make the following prediction: removing the time-censoring bias from observed generation intervals will give us a corrected distribution that contains information about the population structure; using this corrected distribution should allow us to correctly infer the reproductive number from the exponential growth rate. In other words, Trapman \textit{et al.}'s finding can be explained by the gap between the intrinsic generation interval and an ``effective'' (spatially corrected) generation interval.

In this study, we explore spatiotemporal variation in generation intervals measured through contact tracing.
We extend previous frameworks to study how the censored generation-interval distributions change over time and compare this distribution with the backward generation-interval distribution.
We classify sources of spatial variation in generation intervals into three levels (egocentric, local, and global) and discuss how they affect observed generation-interval distributions.
Finally, we compare two methods for accounting for temporal bias and test our prediction using simulations.

\section{Intrinsic generation-interval distributions}

\begin{figure}[!pbth]
\includegraphics[width=\textwidth]{../fig/individual_and_population.pdf}
\caption{\textbf{Comparison of individual- and population-level kernels.}
(Left) an individual-level kernel of an infected individual with latent period of 11.4 days followed by infectious period of 5 days. 
(Right) a population-level kernel of infected individuals with latent and infectious periods exponentially distributed with means of 11.4 and 5 days, respectively. 
Shaded areas under the curves are equal to individual- and population-level reproductive numbers, both of which are set to 2 in this example.
Parameters are chosen to reflect the West African Ebola outbreak \citep{who2014ebola}.}
\label{fig:indpop}
\end{figure}

Generation-interval distributions are often considered as population averages, but we can distinguish population-level distributions from individual-level distributions \citep{svensson2007note, svensson2015influence}; 
making this distinction clear will be particularly useful when we discuss spatial components (see \fref{indpop}).
An individual-level intrinsic infection kernel $k(\tau; a)$ describes the rate at which an infected individual with ``aspect'' $a$ makes ``effective contacts'' (contacts which will cause infection if the contactee is susceptible).
Individual aspects may represent variation in the course of infection (e.g., duration of latent and infectious periods) and the level of infectiousness, which can depend both on biological infectiousness as well as contact patterns.

Assuming that the individual properties are independent of risk of infection, the population-level kernel is given by integrating over these individual variations:
\begin{equation}
K(\tau) = \int k (\tau; a) f(a) da,
\end{equation}
The population-level kernel describes the rate at which secondary infections are expected to be caused by an an infected individual, on \emph{average}.
where $f(a)$ represents a probability density over a (possibly multi-dimensional) aspect space.

Assuming that a population mixes homogeneously, we can write: 
\begin{equation}
K(\tau) = \RR_0 g(\tau),
\end{equation}
where $\RR_0 = \int K(\tau) d\tau$ is the basic reproductive number (the expected number of secondary cases caused by a randomly chosen infectious individual in a fully susceptible population), 
and $g(\tau)$ is the expected time distribution of those cases (the intrinsic generation-interval distribution).

In a homogeneously mixing population, current disease incidence at time $t$, $i(t)$, is the product of the current infectiousness of individuals infected in the past and the current proportion of the population susceptible, $S(t)$.
\begin{equation}
i(t) = S(t) \int K(s) i(t-s) ds = \RR \int g(s) i(t-s) ds,
\end{equation}
where $\RR = S(t) \RR_0$ is the effective reproductive number.
This model, also referred to as the renewal equation, can describe a wide range of epidemic models \citep{heesterbeek1996concept, diekmann2000mathematical, roberts2004modelling, aldis2005integral, wallinga2007generation, roberts2007model}.
Over a period of time where the proportion of susceptible $S$ remains roughly constant, we would expect approximately exponential growth in incidence $i(t)$; assuming $i(t) = i(0) \exp(r t)$ yields the Euler-Lotka equation \citep{lotka1907relation}, which provides a direct link between speed and strength of an epidemic:
\begin{equation}
\frac{1}{\RR} = \int g(s) \exp(-r s) ds.
\end{equation}

\section{Generation-interval distributions across time}

When generation intervals are estimated through contact tracing during an outbreak, infection events that have not happened yet are not observed. 
This effect is called ``right-censoring''.
We can understand the effect of right-censoring using backward generation-interval distributions \citep{tomba2010some, nishiura2010time, champredon2015intrinsic, britton2019estimation}:
the observed (censored) generation-interval distribution is a weighted average of backward generation-interval distributions (weighted by incidence) up until the observation time.

The density of new infections occurring at time $t$ caused by infectors who were infected at time $t-\tau$ is given by
\begin{equation}
i_{t-\tau}(t) = \RR \, i(t-\tau) g(\tau) S(t)
\end{equation}
As shown in \cite{champredon2015intrinsic}, the backward generation-interval distribution, $b_t(\tau)$, describes a distribution of infection that occurred $\tau$ time units before the reference time $t$ and is proportional to $i(t-\tau) g(\tau)$:
\begin{equation}\label{eq:backward}
b_t(\tau) = \frac{i(t-\tau) g(\tau)}{\int_0^t i(t-\sigma) g(\sigma) d\sigma}
\end{equation}
On the other hand, the censored generation-interval distribution, $c_t(\tau)$, describes a distribution of \emph{all} infections that are $\tau$ time units apart from any cohort infected at time $s$ before $t$: $\int_0^t i_{s-\tau}(s) ds$.
Then, the censored generation-interval distribution is given by normalizing
\begin{equation}\label{eq:obsg}
\begin{aligned}
c_t(\tau) 
&= \frac{\int_0^t i_{s-\tau}(s) ds}{\int_0^t \int_0^t i_{s-\sigma}(s) ds d\sigma}= \frac{\int_0^t i(s-\tau) g(\tau) S(s) ds}{\int_0^t \int_0^t i(s-\sigma) g(x) S(s) ds d\sigma}\\
\end{aligned}
\end{equation}
Substituting \eref{backward} allows us to rewrite the censored generation-interval distribution as a weighted distribution of the backward generation intervals:
\begin{equation}
c_t(\tau) \propto \int_0^t b_s(\tau) i(s) ds.
\end{equation}

\begin{figure}[!pbth]
\includegraphics[width=\textwidth]{../fig/temporal_effect.pdf}
\caption{\textbf{Temporal variation in the mean observed generation interval.}
A deterministic Susceptible-Exposed-Infectious-Recovered (SEIR) model was simulated using Ebola-like parameters \citep{who2014ebola}: mean latent period $1/\sigma = 5 \textrm{ days}$, mean infectious period $1/\gamma = 11.4 \textrm{ days}$, and the basic reproductive number $\RR_0 = 2$. 
The backward and the censored mean generation interval were calculated over the course of an epidemic.
The dotted horizontal line represents the mean intrinsic generation interval.
}
\label{fig:censor}
\end{figure}

For a single outbreak, the observed mean  generation interval through contact tracing will always be shorter than the mean intrinsic generation interval (Figure~\ref{fig:censor}).
There are two reasons for this phenomenon.
First, longer generation intervals are more likely to be missed due to right censoring (and short generation intervals are more likely to be observed).
In particular, when an epidemic is growing exponentially ($i(t) = i(0) \exp(rt)$), 
the censored generation-interval distribution (as well as the backward generation-interval distribution) can be written as the intrinsic generation-interval distribution discounted by the rate of exponential growth \citep{britton2019estimation}:
\begin{equation}
c_0(\tau) = b_0(\tau) \propto g(\tau) \exp(-r\tau),
\label{eq:exp}
\end{equation}
where $c_0$ and $b_0$ represent the ``initial'' censored and backward generation-interval distributions (during the exponential growth phase), respectively.
A deterministic simulation confirms that the censored generation-interval distribution has the same mean as the backward generation-interval distribution during this period (Figure~\ref{fig:censor}).
Second, the decreasing number of susceptibles over the course of an epidemic makes long infections less likely to occur \citep{champredon2015intrinsic}.
As a result, even if contact tracing is performed through an entire epidemic, the mean intrinsic generation interval will be underestimated.
Long backward generation intervals near the end of an epidemic have little effect on the censored generation-interval distributions because they represent only a small portion of an outbreak.
Overall, we expect, naively using the observed generation-interval distribution, to underestimate the reproductive number.

% When a disease is at (or near) endemic equilibrium, the number of susceptibles remains (approximately) constant over time, and we expect the observed generation-interval distribution to be similar to the intrinsic generation-interval distribution.

\section{Generation-interval distributions across space}

Infected individuals may contact the same susceptible individual multiple times, but only the first effective contact gives rise to infection in a given individual (after this, they are no longer susceptible).
Therefore, we expect realized generation intervals to be shorter than intrinsic generation intervals, on average, in a limited contact network.

To explore the effect of multiple contacts on realized generation intervals, we first consider the infection process from an ``egocentric'' point of view, taking into account infectious contacts made by a single infector.
We define the egocentric kernel as the rate at which secondary infections are realized by a single primary case with aspect $a$ in the absence of other infectors:
\begin{equation}
\hat{k}(\tau; a) = k(\tau; a) \exp \left(- \delta(a) \int_0^\tau k(s; a) ds\right),
\end{equation}
where $k(\tau; a)$ is the individual-level intrinsic kernel and $e^{- \delta(a) \int_0^\tau k(s; a) ds}$ is the probability that a susceptible acquaintance has not yet been contacted by a particular infected individual.
The dilution term, $\delta(a)$, models how contacts are distributed among susceptible acquaintances.

Throughout this paper, we assume that there is a constant per-pair contact rate $\lambda$.
In this case, the infectiousness of an individual $\RR(a) = \int k(s; a) ds$ is the product of the number of acquaintances $N(a)$ and the contact rate $\lambda$; the dilution term is equal to the reciprocal of the number of acquaintances: $\delta(a) = 1/N(a)$.
This assumption can be relaxed by allowing for asymmetry \citep{trapman2016inferring} or heterogeneity \citep{ball1997epidemics, ball2002general} in contact rates; 
for brevity, we do not pursue these directions here.

The population-level egocentric kernel is found by integrating over individual variations:
\begin{equation}\label{eq:ego}
\hat{K}(\tau) = \int \hat{k}(\tau; a) \hat{f}(a) da,
\end{equation}
where $\hat{f}(a)$ represents a probability density over an aspect space, including the contact structure from an egocentric point of view (i.e., the degree distribution across network). \swp{New notation (f hat)}\jd{This feels like a mistake. We have the same population, and therefore the same distribution over aspect space, even if we are using more information to calculate the egocentric kernel.}
\cite{trapman2016inferring} used this same kernel (also assuming a constant per-pair contact rate) to study the effect of network structure on the estimate of reproductive number.
The population-level egocentric generation-interval distribution is:
\begin{equation}
\hat{g}(\tau) = \frac{\hat{K}(\tau)}{\int \hat{K}(s) ds}.
\label{eq:conditional}
\end{equation}
The population-level egocentric generation-interval distribution describes the distribution of times at which secondary infections are realized by an \emph{average} primary case; for convenience, we will often omit ``population-level''.
Finally, the link between the growth rate and the egocentric reproductive number can be obtained by applying the Euler-Lotka equation to the egocentric generation-interval distribution \citep{trapman2016inferring}:
\begin{equation}
\frac{1}{\hat{\RR}} = \int \hat{g}(\tau) \exp(-r \tau) d\tau.
\label{eq:egorR}
\end{equation}
As the egocentric distribution always has a shorter mean than the intrinsic distribution, \Rhat\ will always be smaller than $\RR$ estimated from the intrinsic distribution;
this generation-interval-based argument provides a clear biological interpretation for the result presented by \cite{trapman2016inferring}.
% Similarly, \Rhat\ will be greater than the \emph{true} reproductive number \RR\, since it does not account for depletion of susceptibles by other routes.

For example, consider a susceptible-exposed-infected-recovered (SEIR) model, which assumes that latent and infectious periods are exponentially distributed.
The intrinsic generation-interval distribution that corresponds to this model can be written as \citep{Champredon2018equivalence}:
\begin{equation}
g(\tau) = \frac{\sigma \gamma}{\sigma - \gamma} \left(e^{-\gamma t} - e^{-\sigma t}\right),
\end{equation}
where $1/\sigma$ and $1/\gamma$ are mean latent and infectious periods, respectively.
Assuming a constant per-pair contact rate of $\lambda$ for any pair, we obtain the following egocentric generation-interval distribution:
\begin{equation}
\hat{g}(\tau) = \frac{\sigma (\gamma + \lambda)}{\sigma - (\gamma + \lambda)} \left(e^{-(\gamma + \lambda)\tau} - e^{-\sigma \tau}\right)
\end{equation}
In this case, with fixed infectiousness during the infection period, the effect of accounting for pairwise contacts is the same as an increase in the recovery rate (by the amount of the per-pair contact rate): infecting a susceptible contact is analogous to no longer being infectious (since the contact cannot be infected again); therefore, the resulting egocentric generation-interval distribution is equivalent to the intrinsic generation-interval distribution with mean latent period of $1/\sigma$ and mean infectious period of $1/(\gamma + \lambda)$.
In practice, directly using this distribution to link $r$ and $\RR$ using the Euler-Lotka equation is unrealistic because it requires that we know the per-pair contact rate. 
Instead, the per-pair contact rate can be inferred from the growth rate $r$, assuming that mean and variance of the degree distribution of a network is known (see \cite{trapman2016inferring} supplementary material section 1.4.2);
we briefly describe this relationship in Section~\ref{egosection}.
\swp{Is it OK to mention this here? Or should I say something about it at the end of the previous paragraph as well? I feel like it's too abstract to say anything there.} \jd{Seems fine this way, but let's discuss your concern.}

This calculation can be validated by simulating stochastic infection processes on a ``star'' network (i.e, a single infected individual at the center connected to multiple susceptible individuals who are not connected with each other).
Simulations (\fref{local}) confirm that in this case the distribution of \emph{contact} times matches the intrinsic generation-interval distribution (left panel), while the distribution of realized generation times (i.e., \emph{infection} times) matches the egocentric generation-interval distribution (middle panel).

\begin{figure}[!pbth]
\includegraphics[width=\textwidth]{../fig/local_effect.pdf}
\caption{
\textbf{Spatial effects on realized generation intervals.}
Theoretical distributions and means are shown in color (and are the same in each panel, for reference). Simulated distributions and means are shown in black.
(Left) the intrinsic generation-interval distribution corresponds to all contacts by a focal individual, regardless of whether the individual contacted is susceptible (simulated on a star network).
(Middle) the egocentric generation-interval distribution corresponds to the distribution of all infectious contacts by the focal individual with susceptible individuals, in the case where the focal individual is the only possible infector (simulated on a star network).
(Right) realized generation-interval distributions are shorter than egocentric distributions in general, because contacts can be wasted when susceptibles become infected through other routes (simulated on a homogeneous network).
All figures were generated using 5000 stochastic simulations on a network with 5 nodes (1 infector and 4 susceptibles) with Ebola-like parameters \citep{who2014ebola}:
mean latent period $1/\sigma = 5 \textrm{ days}$ and mean infectious period $1/\gamma = 11.4 \textrm{ days}$. 
Per-pair contact rate $\lambda = 0.25 \textrm{ days}^{-1}$ is chosen to be sufficiently high so that the differences between generation-interval distributions are clear.
Each simulation is run until all individuals are either susceptible or has recovered.
}
\label{fig:local}
\end{figure}

The egocentric generation interval \eref{conditional} only explains some of the reduction in generation times that occurs on most networks, however.
Generation intervals are also shortened by indirect connections: a susceptible individual can be infected through another route before the focal individual makes infectious contacts.
Simulations on a small homogeneous network confirm this additional effect (\fref{local}, right panel). 

In general, spatial reduction in the mean generation interval can be viewed as an effect of susceptible depletion and can be further classified into three levels: egocentric, local, and global.
Egocentric depletion, as discussed previously, is caused by an infected individual making multiple contacts to the same individual.
Local depletion refers to a depletion of susceptible individuals in a household or neighborhood;
we can think of these structures as small homogeneous networks embedded in a larger population structure (and therefore we can expect similar effects as we see in \fref{local}, right panel).
Both the egocentric and local depletion effects can be observed early in an epidemic, especially in a highly structured population, even if most of the population remains susceptible.
Finally, global depletion refers to overall depeletion of susceptibility at the population level, and explains the reduction in realized compared to intrinsic generation intervals that occurs even in a well-mixed population (\fref{censor}). 

\section{Inferring generation-interval distributions from contact-tracing data}

When generation intervals are sampled through contact tracing, there will be four effects present in the sample: (1) right-censoring effect, (2) egocentric depletion effect, (3) local depletion effect and (4) global depletion effect.
We can correct explicitly for the egocentric effect and, in the case of exponential growth, the right-censoring effect.
While the other two effects are difficult to measure, we can make qualitative predictions about their effects on the realized generation intervals and reproductive numbers: 
both local and global depletion effects reduce number of infections that occur and shorten generation intervals.

Since the right-censoring effect is a sampling bias, we typically want to correct for it. 
In contrast, spatial effects have the same effect on how the epidemic spreads as they do on observed generation intervals. 
We therefore expect that starting from observed generation intervals and correcting for the right-censoring effect, will allow us to estimate an ``effective'' generation interval that accurately reflects dynamics of spread. 
When temporal correction is performed early in an outbreak, during the exponential growth phase, the effective distribution should incorporate egocentric and local spatial effects but not the global effects; we expect this distribution to correctly link $r$ and $\RR$. 
We will call the temporally corrected effective generation-interval distribution $\tsub{g}{eff}(\tau)$.
In a large homogeneously mixing population, the effective distribution is expected to be equivalent to the intrinsic distribution \swp{(since the egocentric depletion is negligible and the local depletion does not exist)}.
\jd{This puts a focus on how we talk about global depletion. It's sort of a thing but also sort of accounted for in the equations that contain $S$. Maybe we should adddress that point explicitly.}

\begin{figure}[!pbth]
\includegraphics[width=\textwidth]{../fig/cmp_reproductive.pdf}
\caption{\textbf{Comparison of estimates of reproductive number based on various methods.}
Using the observed generation-interval distributions (based on the first 1000 generation intervals) without correcting for right-censoring severely underestimates the reproductive number.
Similarly, using the intrinsic generation-interval distribution overestimates the reproductive number because it fails to account for local spatial effects; the egocentric distribution corrects for this only partially.
Both population-level and individual-level methods provide estimates of reproductive number that are consistent with the empirical estimates, which we define as the average number of secondary cases generated by the first 100 infected individuals, but the individual-level method is more precise.
Boxplots are generated using 100 stochastic simulations of the SEIR model on an empirical network using Ebola-like parameters \citep{who2014ebola}: mean latent period $1/\sigma = 5 \textrm{ days}$ and mean infectious period $1/\gamma = 11.4 \textrm{ days}$. 
Per-pair contact rate $\lambda = 0.08 \textrm{ days}^{-1}$ is chosen to be sufficiently high such that differences are clear.
}
\label{fig:cmp}
\end{figure}

Here, we investigate two methods for correcting for temporal bias in contact-tracing data (see Methods for details).
We refer to the first method (Section~\ref{poplevel}) as the population-level method as it relies on the observed distribution aggregated across the entire population.
When an epidemic is growing exponentially, right-censoring causes the observed generation interval to be discounted by the exponential growth rate \eref{exp};
hence, we can ``undo'' the censoring by exponentially weighting the observed generation-interval distribution \citep{tomba2010some, nishiura2010time, britton2019estimation}:
\begin{equation}
\tsub{g}{eff}(\tau) \propto c_t(\tau) \exp(r\tau),
\label{eq:geff}
\end{equation}
where $r$ is the exponential growth rate.

We refer to the second method (Section~\ref{indlevel}) as the individual-level method because it relies on individual contact information.
We model each infection as a non-homogeneous Poisson process arising from the infector \eref{nonhomogeneous}; 
incorporating information about time of infection of an infector, time of infection of an infectee, and time since the beginning of an epidemic allows us to explicitly model the censoring process in the observed generation intervals.
For both methods, mean and coefficient of variation (CV) of generation-interval distributions are estimated by maximum likelihood; the inferred generation-interval distribution is then used to estimate the reproductive number $\RR$ from the observed growth rate $r$ using the Euler-Lotka equation.

To test these methods, we simulate 100 epidemics with Ebola-like parameters on an empirical network \citep{leskovec2016snap}
and compare the estimates of reproductive number with empirical reproductive numbers, which we define as the average number of secondary cases generated by the first 100 infected individuals (\fref{cmp}).
As expected, calculating reproductive number based on the intrinsic generation-interval distribution overestimates the empirical reproductive number;
estimates based on the egocentric generation-interval distribution \eref{conditional} address this problem only partially, as they do not account for indirect (local) spatial effects. 
Direct estimates based on the observed generation intervals from contact tracing severely underestimate the empirical estimates.
While both population- and individual-level corrections provide similar estimates to the empirical reproductive number on average,
population-level estimates are more variable as they are more sensitive to outliers in generation intervals and our estimate of the growth rate.
For smaller values of \RR, we expect the differences to become smaller.
In Appendix, we present the same figure using smaller \RR\ (Appendix A.1) and using Erlang-distributed latent periods (Appendix A.2), which better corresponds to Ebola.
Overall, our qualitative conclusions do not change.

\section{Discussion}

The intrinsic generation-interval distribution, which describes the expected time distribution of secondary cases, provides a direct link between speed (exponential growth rate, $r$) and strength (reproductive number, $\RR$) of an epidemic \citep{wallinga2007generation, svensson2007note, svensson2015influence, park2019practical}.
However, observed generation-interval distributions can vary depending on how and when they are measured \citep{nishiura2010time, tomba2010some, champredon2015intrinsic, britton2019estimation};
determining which distribution correctly links $r$ and $\RR$ can be challenging.
Here, we analyze how observed generation intervals, measured through contact tracing, differ from intrinsic generation intervals.
Changes due to temporal censoring reflect observation bias, whereas changes due to spatial or network structure reflect the dynamics of the outbreak.
Thus, correcting the observed distribution for temporal, but not spatial, effects provides the correct link between $r$ and $\RR$.

Observed generation intervals are subject to right-censoring -- it is not possible to trace individuals that have not been infected yet.
These right-censored distributions can be thought of as averages of ``backward'' generation intervals (measured by tracing the infectors of a cohort of infected individuals) \citep{kenah2008generation, nishiura2010time, tomba2010some, champredon2015intrinsic, britton2019estimation}.
During an ongoing outbreak, the observed generation-interval distribution will always have a shorter mean than the intrinsic-interval distribution due to right-censoring.
Early in the outbreak, censored intervals are expected to match the early-outbreak backward intervals.
Near the end of an outbreak, the effect of right-censoring becomes negligible but the observed generation intervals are still shorter on average than intrinsic generation intervals, because of depletion of the susceptible population.

We think of susceptible depletion as operating on three levels: egocentric, local, and global.
Egocentric susceptible depletion refers to the effect of an infected individual making multiple contacts to the same susceptible individual.
Accounting for the egocentric effect allowed us to link the results by \cite{trapman2016inferring} to established results based on generation intervals.
Local susceptible depletion refers to the effect of other closely related infected individuals (e.g., in a household or neighborhood) making multiple contacts to the same susceptible individual.
Global susceptible depletion refers to the overall decrease in the susceptible pool in the population.

Susceptible depletion happening at all three levels shortens realized generation intervals but acts on different time scales.
Egocentric and local depletion effects are present from the beginning of an epidemic, even when depletion in the global susceptible population is negligible, and can strongly affect the initial spread of an epidemic.
Therefore, we predicted the observed generation intervals during an exponential growth phase to contain information about the contact structure, allowing us to estimate the \emph{effective} generation-interval distribution by simply accounting for the right-censoring.
Simulation studies confirmed our prediction: using the effective generation-interval distribution provides the correct link between $r$ and $\RR$.

We compare two methods for estimating the effective generation-interval distribution and assume that the effective generation-interval distribution follows a gamma distribution.
The gamma approximation of the generation interval distribution has been widely used due to its simplicity \citep{mcbryde2009early, nishiura2009transmission, roberts2011early, trichereau2012estimation, nishiura2015theoretical};
we previously thought that summarizing the entire distribution with two moments (mean and variance) is sufficient to understand the role of generation-interval distributions in linking $r$ and $\RR$ \citep{park2019practical}.
However, further investigation of our methods suggests that making a wrong distributional assumption can lead to biased estimates of the mean and CV of a generation-interval distribution (Appendix A.3), even though the gamma distribution ``looks'' indistinguishable from the shape of the intrinsic generation-interval distribution (derived from the SEIR model).
These results are particularly alarming because it is impossible to know the true shape of the generation-interval distribution for real diseases.
Nonetheless, biases in the parameter estimates of a generation-interval distribution may cancel out and give unbiased estimate of the reproductive number (Appendix A.3).

Generation-interval-based approaches to estimating the reproductive number often assume that an epidemic grows exponentially \citep{wearing2005appropriate, wallinga2007generation, roberts2007model, park2019practical}.
In practice, heterogeneity in population structure can lead to subexponential growth \citep{szendroi2004polynomial, chowell2015western, chowell2016growing, chowell2016characterizing, kiskowski2016modeling, viboud2016generalized};
we therefore expect our simulations on an empirical network to be better characterized by subexponential growth models \citep{viboud2016generalized}.
However, our simulations suggest that the initial exponential growth assumption still provides a viable approach for estimating the reproductive number.

Contact tracing provides an effective way of collecting epidemiological data and controlling an outbreak \citep{clarke1998contact, eames2003contact, donnelly2003epidemiological}.
In particular, using tracing information allows us to infer real-time estimates of the time-varying reproductive number \citep{cauchemez2006estimating, hens2012robust, jewell2012enhancing, soetens2018real};
generation-interval distributions, which can be either assumed or estimated, often play a central role in analyzing tracing data.
Our study illustrates that generation intervals measured through contact-tracing data contain information about the underlying contact structure, which can be implicitly reflected in the estimates of the reproductive number;
this perspective can be particularly useful for characterizing an epidemic because detailed information about the contact structure is often unavailable.

The generation-interval distribution is a key, and often under-appreciated, component of disease modeling and forecasting.
Different definitions, and different measurement approaches, produce different estimates of these distributions.
We have shown that estimates based on observed generation intervals (e.g., through contact tracing) differ in predictable ways from estimates based on underlying measures of infectiousness (e.g., from shedding studies).
These predictable differences can arise from temporal effects, egocentric spatial effects, local spatial (or network) effects and population-level effects.
We have shown that correcting observed intervals for temporal effects allows us to estimate a spatially informed ``effective'' distribution, which accurately describes how disease spreads in a population. 
Future studies should carefully consider how measurement influences estimated generation-interval distributions, and how these distributions influence the spread of disease. 

\section{Methods}

\subsection{Deterministic SEIR model}

To study the effects on right-censoring on the observed generation intervals, we use the deterministic Susceptible-Exposed-Infectious-Recovered (SEIR) model.
The SEIR model describes how a disease spreads in a homogeneously mixing population; it assumes that infeceted individuals become infectious after a latent period.
We use a \seminar\ model, which extends the SEIR model to have multiple equivalent stages in the latent and infectious periods. This gives latent and infectious periods with Erlang distributions (gamma distributions with integer shape parameters), which are often more realistic than the exponentially distributed periods in the standard SEIR model \citep{anderson1980spread, bailey1964some}: 
\begin{equation}
\begin{aligned}
\frac{dS}{dt} &= - \beta S \frac{\sum_{k=1}^{n_I} I_k}{N}\\
\frac{dE_1}{dt} &= \beta S \frac{\sum_{k=1}^{n_I} I_k}{N} - n_E \sigma E_1\\
\frac{dE_m}{dt} &= n_E \sigma (E_{m-1} - E_m) && \text{for } m = 2, 3, \dots, n_E\\
\frac{dI_1}{dt} &= n_E \sigma E_m - n_I \gamma I_1\\
\frac{dI_n}{dt} &= n_I \gamma (I_{n-1} I_n) && \text{for } n = 2, 3, \dots, n_I\\
\end{aligned}
\end{equation}
where $S$ is the number of susceptible individuals, $E_m$ is the number of exposed individuals in the $m$-th compartment, $I_n$ is the number of infectious individuals in the $n$-th compartment, and $R$ is the number of recovered individuals.
Parameters of the model are specified as follows: $N$ is the total population size, $\beta$ is the transmission rate, $1/\sigma$ is the mean latent period, $n_E$ is the number of latent compartments, $1/\gamma$ is the mean infectious period, and $n_I$ is the number of infectious compartments.
While we use the SEIR model for simulations in our main text, we show results based on more realistic distributions in Appendix.

\subsection{Stochastic SEIR model}

We develop an algorithm to simulate an individual-based SEIR model on a network to keep track of generation intervals.
Our algorithm is based on the Gillespie algorithm \citep{gillespie1977exact}.
We model individuals with nodes on a network and their acquaintanceships with edges; infected individuals can only contact their acquaintances.

First, we begin by randomly selecting initially infected individuals; these individuals are assumed to be infected at $t = 0$.
For each infected individual $i$, we randomly draw the latent period $E_i$ from an Erlang distribution with mean $1/\sigma$ and shape $n_E$.
We then construct the random infectious period and infectious contact times simultaneously as follows.
For each of the $n_I$ stages of the infectious period, we draw the number of effective contacts (before transitioning to the next compartment) from a geometric distribution with probability $n_I\gamma/(S_i \lambda + n_I \gamma)$, where $S_i$ is the number
of susceptible acquaintances and $\lambda$ is the per-pair contact rate.
We then choose the time between consecutive events (the chosen number of contacts, followed by exit from the given stage of infection) from an exponential distribution with rate $S_i \lambda + n_I\gamma$.
For each contact, a contactee is uniformly sampled from the set of susceptible acquaintances of the individual $i$.
The infectious period $I_i$ is the sum of all of these waiting times.

After repeating the contact process for all initially infected individuals, all contacts are put into a sorted queue.
The first person in the queue becomes infected, and the current time is updated to infection time of this individual. 
Any subsequent contacts made to this individual are removed from the queue because they will no longer be effective.
We repeat the contact process for this newly infected individual.
Then, new contacts are added to the sorted queue.
The simulation continues until there are no more contacts are left in the queue.

\subsection{Egocentric relationship between $r$ and $\RR$ (SEIR model)}
\label{egosection}

\swp{Can you check if this subsection is clear?}
\jd{Seems clear enough. See framing below.}
Here, we show that the egocentric relationship between $r$ and $\RR$ derived by Trapman \cite{trapman2016inferring} (see the original source for detailed derivations) can be derived simply by applying the Euler-Lotka equation and using the effective (rather than the intrinsic) generation-interval distribution. 
Assume that latent and infectious periods are exponentially distributed with mean $1/\sigma$ and $1/\gamma$, respectively.
Assuming a constant per-pair contact rate of $\lambda$ for any pair, the egocentric generation-interval distribution can be written:
\begin{equation}
\hat{g}(\tau) = \frac{\sigma (\gamma + \lambda)}{\sigma - (\gamma + \lambda)} \left(e^{-(\gamma + \lambda)\tau} - e^{-\sigma \tau}\right).
\end{equation}
Substituting into \eref{egorR}, we get
\begin{equation}
\hat\RR = \left(1 + \frac{r}{\sigma}\right) \left(1 + \frac{r}{\gamma + \lambda}\right),
\end{equation}
where $r$ is the exponential growth rate.
Alternatively, the reproductive number can be expressed based on the degree distribution (mean $\mu$ and variance $\sigma^2$) of a network:
\begin{equation}
\hat\RR = \underbrace{\kappa \lambda}_{\text{average contact rate}} \times \underbrace{\frac{1}{\gamma + \lambda}}_{\text{mean effective infectious period}},
\end{equation}
where $\kappa = \sigma^2/\mu + \mu - 1$, referred to as the mean degree excess \citep{newman2003structure}, describes the expected number of susceptible individuals that an average infected individual will encounter early in an outbreak.
Combining the two equations, we get
\begin{equation}
\lambda = \frac{(\gamma + r) (\sigma + r)}{(\kappa - 1) \sigma - r},
\end{equation}
which completes the relationship between the growth rate and the egocentric reproductive number:
\begin{equation}
\hat\RR = \frac{\gamma + r}{\gamma \sigma/(\sigma + r) + r/\kappa}.
\end{equation}

\subsection{Population-level method for estimating generation-interval distributions from contact-tracing data}
\label{poplevel}

Population-level method tries to estimate the effective generation-interval distribution by reverting the inverse exponential weighting in the observed generation-interval distribution:
\begin{equation}
\tsub{g}{eff}(\tau) \propto c_t(\tau) \exp(r\tau).
\end{equation}
In order to do so, we first fit a gamma distribution to the observed generation-interval distributions; specifically, we estimate the mean $\bar G$ and shape $\alpha$ of a gamma distribution by maximum likelihood.
Then, the effective generation-interval distribution also follows a gamma distribution with mean $\alpha/(\alpha/\bar G - r)$ and shape $\alpha$.

\subsection{Individual-level method for estimating generation-interval distributions from contact-tracing data}
\label{indlevel}

We model each infection $i$ from an infected individual $j$ as a non-homogeneous Poisson process between the time at which infector $j$ was infected ($t_j$) and the censorship time ($\tsub{t}{censor}$), where the time-varying Poisson rate at time $t$ is equal to $\RR \tsub{g}{eff}(t - t_j)$, where $\tsub{g}{eff}(t)$ is the effective generation-interval distribution \citep{daley2007introduction}.
We use a gamma distribution (parameterized by its mean and shape) to model the effective generation-interval distribution.
Then, the probability that an individual $j$ infects $n_j$ individuals between $t_j$ and $\tsub{t}{censor}$ is equal to
\begin{equation}
\frac{\RR^{n_j} \exp \big(- \RR\, \tsub{G}{eff}(\tsub{t}{censor} - t_j ;\theta) \big) \prod_{i=1}^{n_j} \tsub{g}{eff}(s_{i, j}; \theta)}{n_j!},
\end{equation}
where $s_{i,j}$ is the observed generation interval between infector $j$ and infectee $i$, and $\theta$ is a (vector) parameter of the generation-interval distribution $g$ (and the corresponding cumulative distribution function $\tsub{G}{eff}$).

The entire data likelihood of the non-homogeneous poisson process can be written as:
\begin{equation}\label{eq:nonhomogeneous}
\begin{aligned}
&\mathcal{L}(\RR, \theta\,|\, \mathbf{s}, \mathbf{t}, \mathbf{n}, \tsub{t}{censor})\\
&\quad=\prod_{j=1}^N \left(\RR^{n_j} \exp \big(- \RR \, \tsub{G}{eff}(\tsub{t}{censor} - t_j ;\theta) \big) \prod_{i=1}^{n_j} \tsub{g}{eff}(s_{i, j}; \theta) \right),
\end{aligned}
\end{equation}
where $N$ is the total number of infected individuals.
Here, we estimate parameters $\RR$ and $\theta$ by maximum likelihood.
Since the estimate of $\RR$ is sensitive to under-reporting, we use the estimated generation-interval distribution to infer the effective reproductive number from the estimated growth rate (using the Euler-Lotka equation).
\cite{forsberg2008likelihood} proposed a related approach based on discretized incidence reports.

\subsection{Measuring the exponential growth rate}

We estimate the exponential growth rate $r$ of an epidemic from daily incidence by modeling the cumulative incidence $c(t)$ with a logistic function \citep{ma2014estimating}: 
\begin{equation}
c(t) = \frac{K}{1 + \left[(K/c_0) - 1\right] e^{-rt}},
\end{equation}
where $K$ is the final size of an epidemic and $c_0$ is the initial number of cases.
Fitting this curve directly to cumulative incidence can lead to overly confident result \citep{king2015avoidable}; instead we fit interval incidence $x(t) = c(t + \Delta t) - c(t)$, where $\Delta t$ is 1 day, to daily incidence, assuming that daily incidence follows a negative binomial distribution. 
This assumption requires us to estimate the dispersion parameter $\theta$ of a negative binomial distribution.
Parameters $r$, $K$, $c_0$, and $\theta$ are estimated by maximum likelihood.
Fitting time window is defined from the last trough before the peak of an epidemic to the first day after the peak of an epidemic.

\subsection{Empirical network}

To simulate epidemics on a realistic network, we use the `condensed matter physics' network from the Stanford Large Network Dataset Collection \citep{leskovec2016snap}.
This graph describes co-authorship among anyone who submitted a paper to Condensed Matter category in the arXiv between January 1993 and April 2003 \citep{leskovec2007graph}.
It consists of 23133 nodes and 93497 edges.
The same network was used by \cite{trapman2016inferring} to study how network structure affects the estimate of the reproductive number.

\section*{Authors' contributions}

SWP led the literature review, wrote the first draft of the MS, performed analytic calculations and simulations; JD conceived the study and performed analytic calculations; all authors contributed to refining study design, literature review, and final MS writing. All authors gave final approval for publication. 

\section*{Competing interests}

The authors declare that they have no competing interests.

\section*{Funding}

This work was supported by the Canadian Institutes of Health Research [funding reference number 143486].


\pagebreak
\appendix
\renewcommand\thefigure{\thesection.\arabic{figure}}
\setcounter{figure}{0}    
\section{Appendix}

\subsection{Comparison of estimates of reproductive number -- smaller $\RR$}

\begin{figure}[!h]
\includegraphics[width=\textwidth]{../fig/cmp_reproductive_small.pdf}
\caption{\textbf{Comparison of estimates of reproductive number based on various methods.}
This figure matches \fref{cmp} in the main text.
A smaller per-pair contact rate ($\lambda = 0.0026 \textrm{ days}^{-1}$) is used to simulate epidemics.
All other parameters are the same as in \fref{cmp}.
}
\label{fig:cmpsmall}
\end{figure}

\pagebreak

\subsection{Comparison of estimates of reproductive number -- Erlang distributed latent periods}

\begin{figure}[!h]
\includegraphics[width=\textwidth]{../fig/cmp_reproductive_seminr.pdf}
\caption{\textbf{Comparison of estimates of reproductive number based on various methods.}
This figure matches \fref{cmp} in the main text which used a SEIR model.
Here, an Erland-distributed latent period ($n_E=2$) is used to simulate epidemics; this assumption better matches the incubation period distribution of the Ebola virus disease than the exponential assumption \citep{who2014ebola}.
All other parameters are the same as in \fref{cmp}.
}
\label{fig:cmpseminir}
\end{figure}

\pagebreak

\subsection{Testing the individual-based method on simulations on a homogeneous network}

\begin{figure}[!h]
\includegraphics[width=\textwidth]{../fig/full_coverage_fig.pdf}
\caption{\textbf{Wrong distributional assumption may result in biased estimates of the parameters of a generation-interval distribution.}
We simulate a stochastic SEIR model on a homogenous network with $10^5$ individuals using Ebola-like parameters \citep{who2014ebola}: mean latent period $1/\sigma = 5 \textrm{ days}$, mean infectious period $1/\gamma = 11.4 \textrm{ days}$, and the basic reproductive number $\RR_0 = 2$. Then, we apply the individual-based method based on the first 1000 infections.
(A-B) A boxplot of 100 estimates of the mean generation intervals and their coefficient of variations (CV). 
Dashed horizontal lines represent the true value.
(C) Coverage probability of the mean generation intervals and their CVs based on the 95\% confidence interval.
Dashed horizontal lines represent the 95\% coverage probability.
(D) Estimated generation-interval distributions based on the gamma approximation and a true generation-interval distribution of an SEIR model.
(E) Estimated $r$--$\RR$ relationships based on the gamma approximation and a true $r$-$\RR$ relationship of an SEIR model.
Even though there are biases in the estimates of the parameters of the generation-interval distribution, we can still get an unbiased estimate of the $r$--$\RR$ relationship: a shorter mean generation interval decreases $\RR$ whereas a tighter generation-interval distribution increases $\RR$ \citep{wallinga2007generation, park2019practical}.
}
\label{fig:cover}
\end{figure}

\pagebreak

\bibliography{network}
\end{document}
