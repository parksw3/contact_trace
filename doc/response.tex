\documentclass[12pt]{article}
\usepackage[utf8]{inputenc}

\usepackage{color}

\usepackage{natbib}
\setcitestyle{numbers} 
\setcitestyle{square}
\bibliographystyle{prsb}

\usepackage{lmodern}
\usepackage{amssymb,amsmath}

\newcommand{\rR}{\mbox{$r$--$\cal R$}}
\newcommand{\RR}{\ensuremath{{\cal R}}}
\newcommand{\RRhat}{\ensuremath{{\hat \cal R}}}
\newcommand{\Rx}[1]{\ensuremath{{\cal R}_{#1}}} 
\newcommand{\Ro}{\Rx{0}}
\newcommand{\Reff}{\Rx{\mathit{eff}}}
\newcommand{\Tc}{\ensuremath{C}}

\newcommand{\rev}{\subsection*}
\newcommand{\revtext}{\textsf}
\setlength{\parskip}{\baselineskip}
\setlength{\parindent}{0em}

\newcommand{\comment}[3]{\textcolor{#1}{\textbf{[#2: }\textsl{#3}\textbf{]}}}
\newcommand{\jd}[1]{\comment{cyan}{JD}{#1}}
\newcommand{\swp}[1]{\comment{magenta}{SWP}{#1}}
\newcommand{\dc}[1]{\comment{blue}{DC}{#1}}
\newcommand{\jsw}[1]{\comment{green}{JSW}{#1}}
\newcommand{\hotcomment}[1]{\comment{red}{HOT}{#1}}

\begin{document}

\noindent Dear Editor:

Thank you for the chance to revise and resubmit.
We apologize for the delay.
Reviewer 1 suggested that we should divide the paper into two: theory and application.
Given the relevance of this manuscript to the current COVID-19 outbreak, we think it is best to write an application-based one first (this one) and then write a theory-based on later.
We believe that incorporating some theory with application will help readers understand the material better at the moment.
Otherwise, we have tried our best to incorporate the reviewers' suggestions.
See below for detailed response.

\rev{Reviewer \#1}

\revtext{The main objective of this paper is to correct selection bias in observational
(contact tracing) data to yield unbiased estimate of the "true" distribution
corresponding to generation intervals defined in Lines 42-43 (if I understand
properly).}

The main objective is to estimate the effective generation-interval distribution, which implicitly accounts for spatial structure. We tried to make this clear now.

\revtext{Lines 42-43: "A generation interval is defined as the time between when a
person becomes infected and when that person infects another person."
(with reference [10]). Indeed, most of the cited literature in this paper
start with the definition exactly the same way. I feel that this definition
is imprecise and is not as obvious as it reads, especially in its relationship
with the intrinsic generation interval (Line 46) and its distribution (Lines
49,101). It is not proven that they are equivalent.}

We agree that this definition may be imprecise but we feel that it may be clearer for the readers to begin with a familiar definition. 
Instead, we now point to imprecision of this definition.
We also make clear that this definition corresponds to the definition at the individual-level (i.e., a single interval) and distinguish it from the population-level distributions:

``At the individual level, a generation \emph{interval} is defined as the time between when a person becomes infected and when that person infects another person \citep{svensson2007note};
while this definition is widely used in the literature, it is not connected directly to a population-level distribution.''

\revtext{Line 46: "...intrinsic generation intervals..." This is not yet defined. Is it
equivalent to the definition in Lines 42-43?}

We now define intrinsic generation interval properly:

``\emph{intrinsic} generation-interval distributions describe the expected time between infection of a primary case and infection of its secondary cases in a fully susceptible population...''

\revtext{Line 46: "......intrinsic generation intervals measure the infectiousness of an
infected individual." This statement is vague. I will get back to this phrase
a few pages later in the discussion on "equilibrium distributions" and
suggest ways to make it more precise.}

\revtext{Lines 49,101: "...the intrinsic generation interval distribution..." While the
interval itself is not defined in Line 46 (in my opinion), its distribution is
clarified in Line 101 in relation to equation (2) under homogeneous mixing.
I hereby pose the following questions:}

\revtext{
1. Although $g(\tau) > 0$ satisfies all the properties of a probability density
function (p.d.f.) of a non-negative continues random variable, does that
automatically correspond to a specific random variable relevant to any of
the observable (or conceptually observable) "generation intervals"?
}

We now define $g(\tau)$ as follows: ``Expected time between infection of a primary case and infection of its secondary cases in a fully susceptible population''

\revtext{
2. Is it the p.d.f. corresponding to the definition in Lines 42-43? Proof?
}

Definition in Lines 42-43 does not correspond to the definition of the intrinsic generation-interval distribution. We have tried to make this clearer in the text.

\revtext{
Lines 44-48: "... the observed generation-interval distribution can change...
...observed generation intervals refer to the time between actual infection
events." The actual infection events correspond to the exact moments of
passing the infection between a pair of infected-susceptible individuals.
Are they observable? In most cases, including contact tracing, observable
events are clinical (e.g. onset of symptoms). The relationship between
the interval deÖned by the time between actual infection events and the
interval deÖned by the time between clinical onsets is unclear. At least the
latter depends on the incubation period (of the infectee), in addition to the
latent period and the infectious period (of the infector). The incubation
period is defined as the time from infection to symptom onset, and is
different from the latent period such as in the SEIR model. This paper
does not make such distinction. It assumes that the time between actual
infection events "were observable" in a hypothetical way.
}

\revtext{
Suggestion: Re-phrase "observed generation intervals" or make some further
clarification on this. This is important because naive readers may tend
to take the face value and collect different kinds of observed data and
make inferences, such as that based on clinical events. This is potentially
misleading.
}

We agree that calling it ``observed'' can be misleading. We now refer to them as \emph{realized} intervals.

``There are important distinctions to be made when defining generation-interval \emph{distributions} at the population-level: \emph{intrinsic} generation-interval distributions describe the expected time between infection of a primary case and infection of its secondary cases in a fully susceptible population,
while \emph{realized} generation-interval distributions describe the time between actual infection events over the course of an epidemic.
Due to individual variation in infection time, the realized generation-interval distribution (at the population-level) can change depending on when and how it is defined \citep{svensson2007note, kenah2008generation, nishiura2010time, champredon2015intrinsic}.''

\revtext{Lines 54-55: ``the backward generation interval...'' This is a key measurement in this paper along with the length-bias in relation to data observation scheme. Conceptually, there may also be a "forward" generation
interval. Both concepts were initially raised in Svensson (2007, ref.[21]
in the paper), who pointed out the difference in distribution between the
primary generation time as measured prospectively from the time of the
infection of the infector to the transmission to the infectee, and the secondary generation time as measured retrospectively from the time of the
transmission to the infectee to the infection of the infector.}

We now introduce forward.

\revtext{1. does that mean that your objective is to make unbiased inference on the
"forward", or the primary, generation time?}

Thank you for pointing this out. This is what we want. We have tried to clarify this throughout the manuscript. We no longer use the term ``effective'' generation-interval distribution.

\revtext{2. does the definition in Lines 42-43 correspond to the forward generation
interval?}

No.

\revtext{3. does the definition in Lines 42-43 correspond to the intrinsic generation
interval (Line 46)?}

No.


\revtext{4. does the definition in Lines 42-43 correspond to a random variable with
p.d.f. given by $g(\tau)$?}

No.

\revtext{5. do you imply that the intrinsic generation interval (not defined in Line 46
but distributed according to Line 101) and the forward measure, and the
objective of the unbiased inference, are all equivalent?}

No.




\bibliography{networkprsb}
\end{document}
